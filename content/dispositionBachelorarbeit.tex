%%%%%%%%%%%%%%%%%%%%%%%%%%%%%%%%%%%%%%%%%%%%%%%%%%%%%%%%%%%%%%%%%
%_____________ ___    _____  __      __ 
%\____    /   |   \  /  _  \/  \    /  \  Institute of Applied
%  /     /    ~    \/  /_\  \   \/\/   /  Psychology
% /     /\    Y    /    |    \        /   Zürcher Hochschule 
%/_______ \___|_  /\____|__  /\__/\  /    fuer Angewandte Wissen.
%        \/     \/         \/      \/                           
%%%%%%%%%%%%%%%%%%%%%%%%%%%%%%%%%%%%%%%%%%%%%%%%%%%%%%%%%%%%%%%%%
%
% Project     : Latex Vorlage SemArbeit
% Title       : 
% File        : dispositionBachelorarbeit.tex Rev. 00
% Date        : 11.2013
% Author      : Till J. Ernst
%
%%%%%%%%%%%%%%%%%%%%%%%%%%%%%%%%%%%%%%%%%%%%%%%%%%%%%%%%%%%%%%%%%
\chapter*{Disposition Bachelorarbeit}\label{chap.dispo}
\glsresetall
% Kapitel Ausgangslage
\begin{tabularx}{17cm}{lX}
Name des Schreibenden & Till Ernst \\
Name des Referenten & Prof. Dr. phil. habil. Daniel Süss \\
Titel der Arbeit & Hat Multitasking einen Einfluss auf das eigene Glück? Auswirkung von Medien-Multitasking auf das subjektive Wohlbefinden von Studenten\\
\end{tabularx}
% Fragestellung -------------------------------------------------
\section*{Fragestellung / Annahme}\label{section.fragestellung}
\textbf{Hauptfragestellung:} Beeinflusst die Häufigkeit und die Fähigkeit bezogen auf Medien-Multitasking das subjektive Wohlbefinden von Studenten? 
\par
\textbf{Nebenfragestellung:} Wenden Teilzeit-Studenten aufgrund ihrer Rollenvielfalt ein höheres Mass an Medien-Multitasking gegenüber den Vollzeit-Studenten an? \par
Gemäss aktuellen Studien \cite<e.g.,>{Sanbonmatsu2013, Rosen2013} wird davon ausgegangen, dass Multitasking unter anderem von der Fähigkeit, Ablenkung abzublocken, Überschätzen der eigenen Fähigkeit, Impulsivität, Sensation Seeking oder starke Belohnungs- und Nutzenorientierung abhängig ist, um hier nur einige Faktoren zu nennen. Bisher konnte der direkte Zusammenhang zwischen Multitasking und Wohlbefinden nicht nachgewiesen werden \cite{Shih2013}. \\
Diese Studie geht davon aus, dass der Einfluss von einem in hohem Mass angewendeten Medien-Multitasking, mit gleichzeitiger Überschätzung der eigenen Fähigkeit, mit einem geringeren subjektiven Wohlbefinden einhergeht.\\
Zudem wird angenommen, dass Studierende mit mehreren Rollen neben dem eigentlichen Studium (z.B.: Familie, Erwerb, etc.) mehr Medien-Multitasking anwenden, da sie sich damit eine Zeitersparnis erhoffen (siehe Hypothesen).\\ 
% Hypothese -----------------------------------------------------
\section*{Hypothese (statistische Verfahren)}\label{section.hypothesen}
tbd
\section*{Art der Arbeit}\label{section.artArbeit}
Empirische Bachelorarbeit mit quantitativem Charakter. \\
In erster Linie werden empirische Daten erhoben, die mittels deskriptiver Statistik ausgewertet werden sollen. Es handelt sich hierbei um eine Querschnitts-Studie. 
\section*{Theoretischer Hintergrund / Stand der Forschung}\label{section.forschung}
tbd
\section*{Methode}\label{section.methode}
tbd
\section*{Abgrenzung}\label{section.abgrenzung}
tbd
%\section*{}\label{section.}