%%%%%%%%%%%%%%%%%%%%%%%%%%%%%%%%%%%%%%%%%%%%%%%%%%%%%%%%%%%%%%%%%
%_____________ ___    _____  __      __ 
%\____    /   |   \  /  _  \/  \    /  \  Institute of Applied
%  /     /    ~    \/  /_\  \   \/\/   /  Psychology
% /     /\    Y    /    |    \        /   Zürcher Hochschule 
%/_______ \___|_  /\____|__  /\__/\  /    fuer Angewandte Wissen.
%        \/     \/         \/      \/                           
%%%%%%%%%%%%%%%%%%%%%%%%%%%%%%%%%%%%%%%%%%%%%%%%%%%%%%%%%%%%%%%%%
%
% Project     : Latex Vorlage SemArbeit
% Title       : 
% File        : dispositionBachelorarbeit.tex Rev. 00
% Date        : 11.2013
% Author      : Till J. Ernst
%
%%%%%%%%%%%%%%%%%%%%%%%%%%%%%%%%%%%%%%%%%%%%%%%%%%%%%%%%%%%%%%%%%
\chapter*{Disposition Bachelorarbeit}\label{chap.dispo}
\glsresetall
% Kapitel Ausgangslage
\begin{tabularx}{17cm}{lX}
Name des Schreibenden & Till Ernst \\
Name des Referenten & Prof. Dr. phil. habil. Daniel Süss \\
Titel der Arbeit & Hat Multitasking einen Einfluss auf das eigene Glück? Auswirkung von Medien-Multitasking auf das subjektive Wohlbefinden von Studenten\\
\end{tabularx}
% Fragestellung -------------------------------------------------
\section*{Fragestellung / Annahme}\label{section.fragestellung}
\textbf{Hauptfragestellung:} Beeinflusst die Häufigkeit und die Fähigkeit bezogen auf Medien-Multitasking das subjektive Wohlbefinden von Studenten? 
\par
\textbf{Nebenfragestellung:} Wenden Teilzeit-Studenten aufgrund ihrer Rollenvielfalt ein höheres Mass an Medien-Multitasking gegenüber den Vollzeit-Studenten an? \par
\textbf{Annahme:} Die Häufigkeit und die Fähigkeit wie Media-Multitasking angewendet wird wirkt sich auf das subjektive Wohlbefinden von Studierenden aus.
Gemäss aktuellen Studien \cite<e.g.,>{Sanbonmatsu2013, Rosen2013} wird davon ausgegangen, dass Multitasking unter anderem von der Fähigkeit, Ablenkung abzublocken, Überschätzen der eigenen Fähigkeit, Impulsivität, Sensation Seeking oder starke Belohnungs- und Nutzenorientierung abhängig ist, um hier nur einige Faktoren zu nennen. Bisher konnte der direkte Zusammenhang zwischen Multitasking und Wohlbefinden nicht nachgewiesen werden \cite{Shih2013}. \\
Diese Studie geht davon aus, dass der Einfluss von einem in hohem Mass angewendeten Medien-Multitasking, mit gleichzeitiger Überschätzung der eigenen Fähigkeit, mit einem geringeren subjektiven Wohlbefinden einhergeht.\\
Zudem wird angenommen, dass Studierende mit mehreren Rollen neben dem eigentlichen Studium (z.B.: Familie, Erwerb, etc.) mehr Medien-Multitasking anwenden, da sie sich damit eine Zeitersparnis erhoffen.\\ 
Die Studienpopulation setzt sich aus Studierenden aus Universitäten und Fachhochschulen zusammen. Der Trend von Media-Multitasking ist stetig am steigen, was unter anderem auf die rasante Entwicklung von neuen Medien und einer sich stetig entwickelnden Technologie zurückzuführen ist \cite{Lee2012}.
% Hypothese -----------------------------------------------------
\section*{Hypothese}\label{section.hypothesen}
\textbf{Haupthypothesen:}
\begin{itemize}
    \item Zwischen der Fähigkeit von Studierenden Media-Multitasking zu betreiben und der Häufigkeit besteht ein negativer Zusammenhang. Je fähiger Studierende im Umgang Media-Multitasking sind, desto weniger werden sie Media-Multitasking anwenden.
    \item Zwischen der Fähigkeit Media-Multitasking zu betreiben und dem subjektiven Wohlbefinden besteht ein positiver Zusammenhang. Fähige Studenten können besser zwischen schädlichem und nützlichen Media-Multitasking unterscheiden und beeinflussen dadurch das eigene subjektive Wohlbefinden.
    \item Zwischen der Häufigkeit von Media-Multitasking zusammen mit der Unfähigkeit, Ablenkung abzublocken besteht ein negativer Zusammenhang mit dem subjektivem Wohlbefinden. Je mehr Media-Multitasking aus Gründen der eigenen Überschätzung und der Unfähigkeit, Ablenkung abzublocken angewendet wird, desto direkter wirkt sich dies auf das SWB aus.
\end{itemize}
\textbf{Arbeitshypothesen:}
\begin{itemize}
    \item Studierende mit einer hohe Rollenvielfalt wenden vermehrt Media-Multitasking an, um damit Rollenkonflikte zu lösen.
    \item tbd
\end{itemize}
\section*{Art der Arbeit}\label{section.artArbeit}
Empirische Bachelorarbeit mit quantitativem Charakter. \\
In erster Linie werden empirische Daten erhoben, die mittels deskriptiver Statistik ausgewertet werden sollen. Es handelt sich hierbei um eine Querschnitts-Studie. 
% Theoretischer Hintergrund --------------------------------------
\section*{Theoretischer Hintergrund / Stand der Forschung}\label{section.forschung}
tbd
% Methode --------------------------------------------------------
\section*{Methode}\label{section.methode}
tbd
% Abgrenzung -----------------------------------------------------
\section*{Abgrenzung}\label{section.abgrenzung}
tbd
%\section*{}\label{section.}