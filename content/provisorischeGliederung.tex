%%%%%%%%%%%%%%%%%%%%%%%%%%%%%%%%%%%%%%%%%%%%%%%%%%%%%%%%%%%%%%%%%
%_____________ ___    _____  __      __ 
%\____    /   |   \  /  _  \/  \    /  \  Institute of Applied
%  /     /    ~    \/  /_\  \   \/\/   /  Psychology
% /     /\    Y    /    |    \        /   Zürcher Hochschule 
%/_______ \___|_  /\____|__  /\__/\  /    fuer Angewandte Wissen.
%        \/     \/         \/      \/                           
%%%%%%%%%%%%%%%%%%%%%%%%%%%%%%%%%%%%%%%%%%%%%%%%%%%%%%%%%%%%%%%%%
%
% Project     : Latex Vorlage SemArbeit
% Title       : 
% File        : provisorischeGliederung.tex Rev. 00
% Date        : 11.2013
% Author      : Till J. Ernst
%
%%%%%%%%%%%%%%%%%%%%%%%%%%%%%%%%%%%%%%%%%%%%%%%%%%%%%%%%%%%%%%%%%
\chapter*{Provisorische Gliederung}\label{chap.gliederung}
\glsresetall
% Abstract
\textbf{Abstract}\par

% Einleitung
\textbf{Einleitung}\par
\begin{enumerate}
    \item Augangslage
    \item Hintergrund, Begründung und Ziel der Studie
    \item Bisherige Forschung
    \item Begriffsbestimmungen / Definitionen
    \item Medien-Multitasking
    \item Subjektives Wohlbefinden
    \item Fragestellung und Hypothesen
\end{enumerate}

% Methode
\textbf{Methode}\par
\begin{enumerate}
    \item Studiendesign
    \begin{enumerate}
        \item Hauptzielparameter
        \item Durchführung der Untersuchung
    \end{enumerate}
    \item Auswahl der Versuchspersonen
    \begin{enumerate}
        \item Rekrutierung
        \item Einschlusskriterien
        \item Ausschlusskriterien
    \end{enumerate}    
    \item Erhebungsinstrumente
    \begin{enumerate}
        \item \textit{Media Use Questionnaire}
        \item \textit{Operation Span Task}
        \item \textit{Fragen zu Lebenszufriedenheit}
    \end{enumerate}
\end{enumerate}

%Ergebnisse
\textbf{Ergebnisse}\par
\begin{enumerate}
    \item Beschreibung der Verfahren 
    \item Darstellung der Ergebnisse
\end{enumerate}

% Diskussion
\textbf{Diskussion}\par
\begin{enumerate}
    \item Beantwortung der Fragestellung
    \item Interpretation
    \item Methodenkritik
    \item Ausblick
\end{enumerate}

% Literaturverzeichnis
\textbf{Literaturverzeichnis}\par

% Anhang
\textbf{Anhang}\par
