%%%%%%%%%%%%%%%%%%%%%%%%%%%%%%%%%%%%%%%%%%%%%%%%%%%%%%%%%%%%%%%%%
%_____________ ___    _____  __      __ 
%\____    /   |   \  /  _  \/  \    /  \  Institute of Applied
%  /     /    ~    \/  /_\  \   \/\/   /  Psychology
% /     /\    Y    /    |    \        /   Zürcher Hochschule 
%/_______ \___|_  /\____|__  /\__/\  /    fuer Angewandte Wissen.
%        \/     \/         \/      \/                           
%%%%%%%%%%%%%%%%%%%%%%%%%%%%%%%%%%%%%%%%%%%%%%%%%%%%%%%%%%%%%%%%%
%
% Project     : Latex Vorlage SemArbeit
% Title       : 
% File        : provisorischeGliederung.tex Rev. 00
% Date        : 11.2013
% Author      : Till J. Ernst
%
%%%%%%%%%%%%%%%%%%%%%%%%%%%%%%%%%%%%%%%%%%%%%%%%%%%%%%%%%%%%%%%%%
\label{chap.gliederung}
\glsresetall

\begin{enumerate}
    % Abstract
    \item Abstract
    % Einleitung
    \item Einleitung
    \begin{enumerate}
        \item Augangslage
        \item Hintergrund, Begründung und Ziel der Studie
        \item Theoretischer Hintergrund
        \begin{enumerate}
            \item Begriffsbestimmungen, Definitionen 
            \item Medien-Multitasking
            \item Subjektives Wohlbefinden
        \end{enumerate}
        \item Bisherige Forschung
        \item Fazit und Forschungslücke    
        \item Fragestellung und Hypothesen
    \end{enumerate}
    % Methode
    \item Methode
    \begin{enumerate}
        \item Studiendesign
        \begin{enumerate}
            \item Hauptzielparameter
            \item Durchführung der Untersuchung
        \end{enumerate}
        \item Auswahl der Versuchspersonen
        \begin{enumerate}
            \item Rekrutierung
            \item Einschlusskriterien
            \item Ausschlusskriterien
        \end{enumerate}    
        \item Erhebungsinstrumente
        \begin{enumerate}
            \item \textit{Media Use Questionnaire}
            \item \textit{Operation Span Task}
            \item \textit{Fragen zu Lebenszufriedenheit}
        \end{enumerate}
    \end{enumerate}
    %Ergebnisse
    \item Ergebnisse
    \begin{enumerate}
        \item Beschreibung der Verfahren 
        \item Darstellung der Ergebnisse
    \end{enumerate}
    % Diskussion
    \item Diskussion
    \begin{enumerate}
        \item Beantwortung der Fragestellung
        \item Interpretation
        \item Methodenkritik
        \item Ausblick
    \end{enumerate}
    % Literaturverzeichnis
    \item Literaturverzeichnis
    % Anhang
    \item Anhang
\end{enumerate}
